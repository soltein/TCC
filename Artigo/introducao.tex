\chapter{Introdução}
O mercado de delivery tem crescido de forma significativa no Brasil, o mesmo foi impulsionado ainda mais 
devido a pandemia de COVID-19. 

Segundo a Associação brasileira de Bares e Restaurantes(Abrasel), o setor de alimentação movimentou cerca
de R\$ 200 bilhões em 2019, sendo que destes, 11\% foi representado pelo delivery. Em 2020 com a pandemia, o
mercado cresceu para 20\% do faturamento total do setor\cite{abraselexpansao}.

Durante a pandemia tivemos um grande aumento no uso de plataformas de delivery, principalmente
do ifood\cite{ifoodprincipal} que hoje é a principal plataforma de delivery no Brasil. Essas plataformas possuem
uma taxa altíssima, mesmo as altas taxas aplicadas, que variam de 12\% a 27\%, além de uma
mensalidade em torno de R\$ 100,00\cite{ifoodtaxa}.

De acordo com o Instituto FoodService Brasil(IFB), no Brasil espera-se que o delivery cresça em torno de 7,5\% em 2023,
fazendo com que os estabelecimentos invistam em ampliar e aprimorar este serviço de entrega de comida\cite{ifb2023}.

A transformação digital\cite{transformacaodigital} está presente nas nossas atividades diárias, e não está limitada ao uso pessoal, o setor 
de delivery está cada vez mais preparado digitalmente, e os estabelecimentos que não adotarem as tecnologias que 
possam facilitar seus processos, controles e integração com os clientes ficarão para trás no mercado. 

Com a utilização de tecnologias como sistemas gerenciais, o estabelecimento evita e minimiza erros e prejuízos nos 
seus processos, tem um melhor controle de estoque, financeiro e diversas rotinas do seu comércio. Além de otimizar
o tempo tanto de preparação com as comandas chegando automáticamente via sistema, sem a movimentação de papel que 
pode se perder no processo, quanto a agilidade em atender o cliente e cadastrar seu pedido, algo que pode ser inclusive
feito pelo próprio cliente direto de sua casa.