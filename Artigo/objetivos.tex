\chapter{Objetivo}

O objetivo do projeto é apresentar uma arquitetura de um sistema para estabelecimentos de alimentação, de modo
que os mesmos possam ter um controle dos seus produtos, clientes e da sua rotina. Além de um sistema que permita
aos clientes efetuarem pedidos de forma online.

Portanto, com objetivo de desenvolver a modelagem de um software no qual os estabelecimentos possam utilizar e 
oferecer através do mesmo diversos benefícios aos seus clientes, como promoções, fidelização do cliente,
mensagens diretas ao smartphone do cliente, até mesmo preços mais baixos desde que o estabelecimento faça um bom marketing para 
atrair clientes que utilizem plataformas de pedidos como ifood.

O cliente poderá acessar o software através de um aplicativo para smartphone, seja uma progressive
web page, ou mesmo um aplicativo nativo, o que será definido no decorrer das análises e projeto arquitetural.

\section{Objetivos Específicos}
Os objetivos específicos são:

\begin{itemize}
    \item Permitir ao estabelecimento ter um controle dos seus cadastros de clientes, produtos, promoções, etc.
    \item Oferecer ao cliente do estabelecimento um sistema facilitador para que o mesmo possa efetuar pedidos diretamente da sua casa.
    \item Facilitar ao estabelecimento oferecer promoções diretamente ao cliente em casa, através do próprio sistema, isto é, enviar promoções para o cliente, 
    mensagens de aniversário ou que mais convier ao estabelecimento.
    \item Facilitar o controle de estoque do estabelecimento. 
    \item Facilitar o controle financeiro do estabelecimento.
    \item Permitir o fácil gerenciamento do estabelecimento através de diversos relatorios.
    \item Oferecer ao cliente a opção de pagar online, para isso deverá ser feito integrações com os principais players de pagamento, como pagseguro, mercadopago, etc.
    \item Permitir ao estabelecimento receber pedidos de plataformas de delivery como ifood.
\end{itemize}

