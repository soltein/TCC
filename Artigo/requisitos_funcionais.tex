\pgfmathsetmacro\nreq{0}
\chapter{Requisitos Funcionais}
Abaixo listamos os requisitos funcionais da aplicação\cite{Sommerville:2011}.
Os requisitos terão níveis de dificuldade e prioridades definidos em B - Baixa, M - Média e A - Alta.

\section{Módulo de produtos}
\vspace{0.5cm}

\pgfmathtruncatemacro\nreq{\nreq + 1}
\subsection{RF-\nreq: Cadastro de produtos}
O sistema deve permitir que o colaborador cadastre produtos com nome, categoria, preço de custo por tamanho, preço de venda por tamanho, 
unidade de medida, descrição, foto, se será feito controle de estoque, ficha técnica, complementos.

\vspace{0.5cm}
\noindent\textbf{\textit{Dificuldade:}} \textit{M -} \textbf{\textit{Prioridade:}} \textit{A}
\vspace{0.5cm}
\pgfmathtruncatemacro\nreq{\nreq + 1}
\subsection{RF-\nreq: Cadastro de categoria}
O sistema deve permitir cadastrar a categoria de produtos, definindo possiveis tamanhos dos produtos nessa categoria ou tamanho únicoa, por exemplo: Refrigerantes, definir os 
tamanhos 350ml, 500ml, 1l, 1.5l, 2l, etc.

\vspace{0.5cm}
\noindent\textbf{\textit{Dificuldade:}} \textit{M -} \textbf{\textit{Prioridade:}} \textit{A}
\vspace{0.5cm}
\pgfmathtruncatemacro\nreq{\nreq + 1}
\subsection{RF-\nreq: Cadastro de complementos}
O sistema deverá permitir cadastrar complementos e seus valores, os mesmos deverão ser ligados ao tamanho do produto no cadastro de produtos.
Exemplo: Acrescimo de Bacon pizza M, Acrescimo de Bacon pizza G e assim por diante.

\vspace{0.5cm}
\noindent\textbf{\textit{Dificuldade:}} \textit{B -} \textbf{\textit{Prioridade:}} \textit{M}
\vspace{0.5cm}
\pgfmathtruncatemacro\nreq{\nreq + 1}
\subsection{RF-\nreq: Cadastro de combos}
O sistema deverá permitir o cadastro de combos, juntando vários produtos cadastrados em um combo e definindo o valor
para o combo.

\vspace{0.5cm}
\noindent\textbf{\textit{Dificuldade:}} \textit{A -} \textbf{\textit{Prioridade:}} \textit{M}
\vspace{0.5cm}

\section{Módulo de clientes}
\vspace{0.5cm}
\pgfmathtruncatemacro\nreq{\nreq + 1}
\subsection{RF-\nreq: Cadastro de clientes}
O sistema deve permitir cadastrar os clientes com nome, email, data nascimento, telefone principal, cpf ou cnpj, identidade ou 
inscrição estadual, endereço principal(endereço, número, complemento, bairro, cidade, cep, valor do frete).

\vspace{0.5cm}
\noindent\textbf{\textit{Dificuldade:}} \textit{B -} \textbf{\textit{Prioridade:}} \textit{A}
\vspace{0.5cm}

\pgfmathtruncatemacro\nreq{\nreq + 1}
\subsection{RF-\nreq: Cadastro de endereços de clientes}
O sistema deverá permitir cadastrar mais endereços para o cliente, com nome do endereço e os campos definidos no requisito RF-004.

\vspace{0.5cm}
\noindent\textbf{\textit{Dificuldade:}} \textit{B -} \textbf{\textit{Prioridade:}} \textit{M}
\vspace{0.5cm}

\section{Módulo de pedidos}
\vspace{0.5cm}
\pgfmathtruncatemacro\nreq{\nreq + 1}
\subsection{RF-\nreq: Cadastro de pedidos delivery}
O sistema deve permitir cadastrar pedidos para os clientes, selecionando cliente, produtos, endereço de entrega,
forma de pagamento, status do pedido(Aberto, Em preparo, Pronto/Para retirar, Saiu para Entregar, Finalizado/Entregue),
lançamento de desconto ou taxa de serviço.

\vspace{0.5cm}
\noindent\textbf{\textit{Dificuldade:}} \textit{A -} \textbf{\textit{Prioridade:}} \textit{A}
\vspace{0.5cm}

\pgfmathtruncatemacro\nreq{\nreq + 1}
\subsection{RF-\nreq: Cadastro de pedidos mesa}
O sistema deve permitir cadastrar pedidos para os clientes no estabelecimento, selecionando cliente, produtos, 
forma de pagamento, status do pedido(Aberto, Em preparo, Pronto/Para retirar, Saiu para Entregar, Finalizado/Entregue),
lançamento de desconto ou taxa de serviço, divisão de valores.

\vspace{0.5cm}
\noindent\textbf{\textit{Dificuldade:}} \textit{A -} \textbf{\textit{Prioridade:}} \textit{A}
\vspace{0.5cm}

\pgfmathtruncatemacro\nreq{\nreq + 1}
\subsection{RF-\nreq: Cadastro de pedidos cliente}
O sistema deve permitir que o próprio cliente faça seu pedido online, selecionando os produtos, endereço de entrega.

\vspace{0.5cm}
\noindent\textbf{\textit{Dificuldade:}} \textit{A -} \textbf{\textit{Prioridade:}} \textit{A}
\vspace{0.5cm}

\section{Módulo de estoque}
\vspace{0.5cm}
\pgfmathtruncatemacro\nreq{\nreq + 1}
\subsection{RF-\nreq: Cadastro de ingredientes}
O sistema deve permitir o gerenciamento do estoque através do cadastro de ingredientes, e nos produtos que 
controlam estoque ter a ficha técnica dos mesmos ligadas aos ingredientes. O cadastro deverá ter informações
como nome do ingrediente, unidade de medida, data de vencimento, preço de custo.

\vspace{0.5cm}
\noindent\textbf{\textit{Dificuldade:}} \textit{A -} \textbf{\textit{Prioridade:}} \textit{M}
\vspace{0.5cm}
\pgfmathtruncatemacro\nreq{\nreq + 1}
\subsection{RF-\nreq: Relatório de estoque}
O sistema deve possuir relatorio que mostre o estoque de produtos e os produtos com vencimento próximo, de modo a 
evitar perdas.

\vspace{0.5cm}
\noindent\textbf{\textit{Dificuldade:}} \textit{A -} \textbf{\textit{Prioridade:}} \textit{M}
\vspace{0.5cm}

\section{Módulo financeiro}
\vspace{0.5cm}
\pgfmathtruncatemacro\nreq{\nreq + 1}
\subsection{RF-\nreq: Formas de pagamento}
O sistema deve permitir o gerenciamento das formas de pagamento, para que possam ser selecionadas no lançamento do
pedido. Exemplo: Cartão de Credito, Cartão de Débito, Dinheiro, Pix, etc. Também as respectivas bandeiras quando 
necessário.

\vspace{0.5cm}
\noindent\textbf{\textit{Dificuldade:}} \textit{A -} \textbf{\textit{Prioridade:}} \textit{M}
\vspace{0.5cm}
\pgfmathtruncatemacro\nreq{\nreq + 1}
\subsection{RF-\nreq: Pagamento Online}
O sistema deve fazer integração com pelo menos 1 meio de pagamento online como pagseguro ou mercado pago.

\vspace{0.5cm}
\noindent\textbf{\textit{Dificuldade:}} \textit{A -} \textbf{\textit{Prioridade:}} \textit{B}
\vspace{0.5cm}

\section{Módulo autenticação/autorização}
\vspace{0.5cm}
\pgfmathtruncatemacro\nreq{\nreq + 1}
\subsection{RF-\nreq: Níveis de acesso}
O sistema deve permitir o cadastro/gerenciamento de níveis de acesso, a principio dois níveis serão suficientes: Usuário e 
Administrador.
Os usuários poderão fazer lançamento de pedidos, cadastros de clientes.
Os administradores terão acesso irrestrito ao sistema.

\vspace{0.5cm}
\noindent\textbf{\textit{Dificuldade:}} \textit{B -} \textbf{\textit{Prioridade:}} \textit{A}
\vspace{0.5cm}

\pgfmathtruncatemacro\nreq{\nreq + 1}
\subsection{RF-\nreq: Autenticação}
O sistema deve ser fechado, não permitido o acesso a ele, somente para usuários cadastrados através de autenticação.
Clientes externos deverão poder acessar um cardápio online e ao efetuar pedido fazer o devido cadastro.

\vspace{0.5cm}
\noindent\textbf{\textit{Dificuldade:}} \textit{A -} \textbf{\textit{Prioridade:}} \textit{A}
\vspace{0.5cm}

